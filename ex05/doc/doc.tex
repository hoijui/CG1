\documentclass[a4paper,headings=small]{scrartcl}
\KOMAoptions{DIV=12}

\usepackage[utf8x]{inputenc}
\usepackage{amsmath}
\usepackage{graphicx}
\usepackage{multirow}
\usepackage{listings}

% define style of numbering
\numberwithin{equation}{section} % use separate numbering per section
\numberwithin{figure}{section}   % use separate numbering per section

% instead of using indents to denote a new paragraph, we add space before it
\setlength{\parindent}{0pt}
\setlength{\parskip}{10pt plus 1pt minus 1pt}

\title{Excercise 5 - \emph{Trace my Ray}}
\subtitle{Computer Graphics I - WS12/13}
\author{\textbf{Team 6} (Rolf Schröder [340126], Robin Vobruba [343773], Bernd Loeber)}
\date{\today}

\pdfinfo{%
  /Title    (Computer Graphics I - WS12/13 - Excercise 5 - Trace my Ray)
  /Author   (Team 6: Rolf Schröder [340126], Robin Vobruba [343773], Bernd Loeber [346336])
  /Creator  ()
  /Producer ()
  /Subject  ()
  /Keywords ()
}

% Simple picture reference
%   Usage: \image{#1}{#2}{#3}
%     #1: file-name of the image
%     #2: percentual width (decimal)
%     #3: caption/description
%
%   Example:
%     \image{myPicture}{0.8}{My huge house}
%     See fig. \ref{fig:myPicture}.
\newcommand{\image}[3]{
	\begin{figure}[htbp]
		\centering
		\includegraphics[width=#2\textwidth]{#1}
		\caption{#3}
		\label{fig:#1}
	\end{figure}
}


\begin{document}

\maketitle

\subsection*{1. Whitted ray tracing}
% Whitted ray tracing ermöglicht neben spekularer Reflektion auch spekulare Brechung nach Snell’s Gesetz.
% Berechnen Sie den effektiven Brechungswinkel für Transmission durch 10 cm Wasser.
% (0.5 Punkte)

Allgemein:
\begin{align*}
n_1 \sin(\alpha_1) &= n_2 \sin(\alpha_2) \\
\alpha_1 &= \arcsin(\sin(\alpha_2 \frac{n_2}{n_1}))
\end{align*}

Brechunswinkel Luft-Wasser-Luft, siehe fig. \ref{fig:img/brechungswinkel}:
\begin{align*}
n_W &= 1.33 \quad \text{Wasser-Brechungsindizes} \\
n_L &= 1.000292 \quad \text{Luft-Brechungsindizes} \\
\beta
	&= \arcsin(\sin(\alpha) \frac{n_L}{n_W}) \\
	&= \arcsin(\sin(\alpha) \frac{1.000292}{1.33}) \\
\gamma
	&= \arcsin(\sin(\beta) \frac{n_W}{n_L}) \\
	&= \arcsin(\sin(\beta) \frac{1.33}{1.000292}) \\
\end{align*}

Der effektive Brechungswinkel beträgt 0 Grad, da der Strahl mit dem selben Winkel aus dem Wasser austritt mit dem er auf das Wasser getroffen ist.

\image{img/brechungswinkel}{0.8}{Lichtbrechung in verschiedenen Medien}

Quellen:
\begin{itemize}
\item http://de.wikipedia.org/wiki/Brechungsindex
\item http://de.wikipedia.org/wiki/Snelliussches\_Brechungsgesetz
\end{itemize}


\subsection*{2. Radiometrie}

\subsubsection*{a) Leuchtkraft der Sonne}
% Die Leuchtkraft der Sonne beträgt 3.846 × 10^25 Watt für eine Wellenlänge im sichtbaren Bereich.
% Wie groß ist die mittlere Strahlungsdichte der Sonne?
% Begründen Sie Ihre Antwort.
% (0.5 Punkte)

Die Leuchtkraft der Sonne verteilt sich kugelförmig im Raum.
Die mittlere Entfernung zwischen der Erde und der Sonne beträgt $1.496 * 10^{11}$ m.
In dieser Entfernung verteilt sich die Leuchtkraft also auf eine Kugel der Größe $A = 4 * \pi * r^{2}$ mit $r = 1.496 * 10^{11}$ , d.h. $A = 2.812 * 10^{23} m^{2}$.
Die mittlere Strahlungsdichte der Sonne auf der Erde beträgt also $3.846 * 10^{26} / 2.812 * 10^{23} = 1367.78 W / m^{2}$.

\subsubsection*{a) Lichtenergie in Berlin}
% Wie viel Lichtenergie fällt auch eine 1m^2 große Fläche in Berlin in 1 min. am 21. Juni um 12:00 Mittags?
% Fertige hierzu zunächst eine Skizze an.
% (1 Punkt)

\image{img/sonneneinstrahlung2}{0.8}{Sonneneinstrahlung}

Siehe fig. \ref{fig:img/sonneneinstrahlung}.

Die Solarkonstante beträgt $1367 * \frac{W}{m^2}$. Rechnet man diese auf die 60 Sekunden um, so erhält man $82020 * \frac{W s}{m^2}$ bzw. $82020 * \frac{J}{m^2}$.
Da nur ca. $54\%$ der Sonnenenergie am Boden ankommt (s.u. NASA) , beträgt die gemessene Energie rund $44290.8 * \frac{J}{m^2}$, bzw. $738.18 * \frac{W min}{m^2}$.
Hierbei wird davon ausgegangen, dass keine Wolken oder Smog/Feinstaub die Sonne verdecken. Da die Sonne mittags hoch steht, sind auch keine Schatten zu erwarten.
Da Berlin relativ nördlich liegt, ist es wahrscheinlich, dass die $738.18 * \frac{W min}{m^2}$ auf Grund der längeren Strecke eher Richtung $700 * \frac{W min}{m^2}$ tendieren (siehe erster Wikipedia Artikel).

Quellen:
\begin{itemize}
\item http://de.wikipedia.org/wiki/Sonnenstrahlung
\item http://de.wikipedia.org/wiki/Solarkonstante
\item http://de.wikipedia.org/wiki/Sonnenwende
\item http://education.gsfc.nasa.gov/experimental/all98invProject.Site/Pages/trl/inv2-1.abstract.html
\end{itemize}


\subsection*{3. Ray-Tracing \& Radiosity}
% Beschreiben Sie, wie man Ray Tracing und Radiosity geeignet kombinieren kann.
% (1 Punkt)

Ray Tracing nutzt perfekte Reflektionen und Radiosity geht von perfekter Streuung aus (Lambertstrahler), welche es in der Realität nicht, bzw. nur sehr selten gibt.
Man kann beide nutzen um realistische (zu gewissen Anteilen streuende und reflektierende, gerichtet diffuse) Beleuchtung zu erstellen.

Mittels Radiosity ist es möglich diffuses Licht korrekt wiederzugeben. Dabei werden auch Effekte wie Farbbluten, indirekte Beleuchtung und Schattenverläufe korrekt wiedergegeben. Ray Tracing kann spiegelnde und transparente Flächen korrekt wiedergeben.


\subsection*{4. Bidirectional path tracing}
% Bidirectional path tracing wurde von Veach Guibas und Lafortune Willems eingeführt.
% Diskutieren Sie Vor- und Nachteile des Verfahrens.
% (1 Punkte)

\textbf{Vorteile}
\begin{itemize}
\item Das Integral konvergiert schneller als beim herkömmlichen Path Tracing.
\item Performanzgewinn durch schnellere Konvergenz ist größer als der Aufwand für das Tracen der zwei Strahlen (Shooting und Gathering).
\item Sehr exaktes Verfahren. Wird teilweise für Referenzbilder verwendet.
\item Simuliert Effekte wie weiche Schatten, Kaustiken und Motion Blur ohne Extraaufwand.
\item Basiert nicht auf perfekt diffusen Lambertstrahlern, wie Radiosity, welche es in der Realität nicht gibt.
\end{itemize}

\textbf{Nachteile}
\begin{itemize}
\item Um Störungen (Noise) im Bild zu vermeiden eine hohe Anzahl von Strahlen getracet werden.
\item Obwohl es schneller als normales Path Tracing ist, benötigt Bidirectional Path Tracing viel Rechenzeit.
\item Pefekt spekuläre Reflektionen können nicht dargestellt werden, da nicht unterschieden werden kann, ob der Strahl von der Reflektion oder vom Bidirectional Path Tracing kommt.
\end{itemize}

\end{document}
