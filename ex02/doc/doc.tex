\documentclass[a4paper,headings=small]{scrartcl}
\KOMAoptions{DIV=12}

\usepackage[utf8x]{inputenc}
\usepackage{amsmath}
\usepackage{graphicx}
\usepackage{multirow}
\usepackage{listings}

% define style of numbering
\numberwithin{equation}{section} % use separate numbering per section
\numberwithin{figure}{section}   % use separate numbering per section

% instead of using indents to denote a new paragraph, we add space before it
\setlength{\parindent}{0pt}
\setlength{\parskip}{10pt plus 1pt minus 1pt}

\title{Excercise 2 - \emph{Visibility}}
\subtitle{Computer Graphics I - WS12/13}
\author{\textbf{Team 6} (Eugen Torkin, Rolf Paul Schroeder, Robin Vobruba)}
\date{\today}

\pdfinfo{%
  /Title    (Computer Graphics I - WS12/13 - Excercise 2 - Visibility)
  /Author   (Team 6: Eugen Torkin, Rolf Paul Schroeder, Robin Vobruba)
  /Creator  ()
  /Producer ()
  /Subject  ()
  /Keywords ()
}

% Simple picture reference
%   Usage: \image{#1}{#2}{#3}
%     #1: file-name of the image
%     #2: percentual width (decimal)
%     #3: caption/description
%
%   Example:
%     \image{myPicture}{0.8}{My huge house}
%     See fig. \ref{fig:myPicture}.
\newcommand{\image}[3]{
	\begin{figure}[htbp]
		\centering
		\includegraphics[width=#2\textwidth]{#1}
		\caption{#3}
		\label{fig:#1}
	\end{figure}
}


\begin{document}

\maketitle


\section{Theorieaufgaben}

\subsection{1}

Bei Clipping muss berechnet werden, ob ein Objekt sich außerhalb oder innerhalb des Sichtvolumens befindet (oder eventuell beides). Dazu werden die Schnittpunkte des Objekts mit dem Sichtvolumen benötigt. Die Berechnung der Schnittpunkte mit dem kanonischen Sichtvolumen (d.h. in einem normierten Würfel parallel zu den Sichtachsen) ist einfacher als diejenige zur Berechnung der Schnittpunkte mit einer Pyramide.

\subsection{2}

Das kanonische Sichtvolumen hat 6 Halbräume (oben, unten, links, rechts, davor, dahinter). Deshalb braucht man einen 6-Bit-Outcode. Wir übernehmen die Codierung für 2D und hängen zwei Ziffern für davor und dahinter hinten an. Das fünfte Bit wird auf 1 gesetzt, wenn sich das Objekt vor dem Sichtvolumen befindet; das sechste Bit wird auf ein gesetzt, wenn sich das Objekt hinter dem Sichtvolumen befindet.

Endpunkte vor dem Sichtvolumen:
\begin{verbatim}
100110 | 100010 | 101010
------------------------
000110 | 000010 | 001010
------------------------
010110 | 010010 | 011010
\end{verbatim}

Endpunkte auf Höhe (z-Achse) des Sichtvolumen:
\begin{verbatim}
100100 | 100000 | 101000
------------------------
000100 | 000000 | 001000
------------------------
010100 | 010000 | 011000
\end{verbatim}

Endpunkte hinter dem Sichtvolumen:
\begin{verbatim}
100101 | 100001 | 101001
------------------------
000101 | 000001 | 001001
------------------------
010101 | 010001 | 011001
\end{verbatim}

\subsection{5}

Im kanonische Sichtvolumen weisen alle Blickstrahlen in dieselbe Richtung. Dadurch gibt es zwischen Normalenvektore eines Polygons und den Blickstrahlen nur genau einen Winkel. Und dann weiter ...

\subsection{6}

\subsubsection{a}

\subsubsection{b}

\subsubsection{c}
Nur othogonale Transformationen erhalten die orthogonalitaet.
Affinie Transformationen sind eine Uebermenge der othogonalen.
Sie ermoeglichen also auch othogonalitaet zerstoerende veraenderungen.
Normalenvektoren muessen aber immer orthogonal zu ihrem objekt blieben.

\subsubsection{d}
Ja, weil man sich jedes Objekt aus unendlich kleinen ebenen zusammengebaut vorstellen kann.


\end{document}

