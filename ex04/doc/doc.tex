\documentclass[a4paper,headings=small]{scrartcl}
\KOMAoptions{DIV=12}

\usepackage[utf8x]{inputenc}
\usepackage{amsmath}
\usepackage{graphicx}
\usepackage{multirow}
\usepackage{listings}

% define style of numbering
\numberwithin{equation}{section} % use separate numbering per section
\numberwithin{figure}{section}   % use separate numbering per section

% instead of using indents to denote a new paragraph, we add space before it
\setlength{\parindent}{0pt}
\setlength{\parskip}{10pt plus 1pt minus 1pt}

\title{Excercise 4 - \emph{Picture me naked}}
\subtitle{Computer Graphics I - WS12/13}
\author{\textbf{Team 6} (Rolf Schröder [340126], Robin Vobruba [343773])}
\date{\today}

\pdfinfo{%
  /Title    (Computer Graphics I - WS12/13 - Excercise 4 - Picture me naked)
  /Author   (Team 6: Rolf Schröder [340126], Robin Vobruba [343773])
  /Creator  ()
  /Producer ()
  /Subject  ()
  /Keywords ()
}

% Simple picture reference
%   Usage: \image{#1}{#2}{#3}
%     #1: file-name of the image
%     #2: percentual width (decimal)
%     #3: caption/description
%
%   Example:
%     \image{myPicture}{0.8}{My huge house}
%     See fig. \ref{fig:myPicture}.
\newcommand{\image}[3]{
	\begin{figure}[htbp]
		\centering
		\includegraphics[width=#2\textwidth]{#1}
		\caption{#3}
		\label{fig:#1}
	\end{figure}
}


\begin{document}

\maketitle

\subsection*{1. Abbildungsfehler}
Um das Dreieck mit einer Textur zu überziehen, muss jedem Pixel des Dreiecks ein Texel der Textur zugeordnet werden.
Wenn die Größe des Dreicks nicht mit der Größe der Textur übereinstimmt, kommt es zwangsläufig zu Abtastfehlern.
Bei einem Dreieck, das größer als die Textur ist, werden nebeneinander liegende Pixel häufig auf dasselbe Texel abgebildet (Pixel und Texel sind immer ganzzahlig).
So kann es passieren, dass ein Texel mehrere Male auf dem Dreieck gezeichnet wird (Oversampling).
Im umgekehrten Fall (Undersampling) ist die Textur wesentlich größer ("detailreicher") als das Dreieck.
Nun werden nicht alle Texel ausgelesen, weil das Dreieck nicht genügend Pixel hat.
Dadurch fehlen Teile der Textur auf dem gerenderten Dreieck.

\subsection*{2. MipMap}

\subsection*{3. Zweischritt-Verfahren}

\subsection*{4. Parametrisierung einer Environment Map}

\subsection*{5. Projektives Texture Mapping}

\end{document}

