\documentclass[a4paper,headings=small]{scrartcl}
\KOMAoptions{DIV=12}

\usepackage[utf8x]{inputenc}
\usepackage{amsmath}
\usepackage{graphicx}
\usepackage{multirow}
\usepackage{listings}

% define style of numbering
\numberwithin{equation}{section} % use separate numbering per section
\numberwithin{figure}{section}   % use separate numbering per section

% instead of using indents to denote a new paragraph, we add space before it
\setlength{\parindent}{0pt}
\setlength{\parskip}{10pt plus 1pt minus 1pt}

\title{Excercise 3 - \emph{Let there be Light}}
\subtitle{Computer Graphics I - WS12/13}
\author{\textbf{Team 6} (Eugen Torkin [341926], Rolf Schröder [340126], Robin Vobruba [343773])}
\date{\today}

\pdfinfo{%
  /Title    (Computer Graphics I - WS12/13 - Excercise 3 - Let there be Light)
  /Author   (Team 6: Eugen Torkin, Rolf Schröder [340126], Robin Vobruba [343773])
  /Creator  ()
  /Producer ()
  /Subject  ()
  /Keywords ()
}

% Simple picture reference
%   Usage: \image{#1}{#2}{#3}
%     #1: file-name of the image
%     #2: percentual width (decimal)
%     #3: caption/description
%
%   Example:
%     \image{myPicture}{0.8}{My huge house}
%     See fig. \ref{fig:myPicture}.
\newcommand{\image}[3]{
	\begin{figure}[htbp]
		\centering
		\includegraphics[width=#2\textwidth]{#1}
		\caption{#3}
		\label{fig:#1}
	\end{figure}
}


\begin{document}

\maketitle


\subsection*{1. Phänomenologie von Beleuchtungsmodellen}
a) Man würde nach wie vor ein Dreieck sehen, bei dem die Kanten heller dargestellt werden als das Dreiecksinnere.
Allerdings ist der Eckpunkt, auf dem sich die Normale befindet, welche in Richting Lichtquelle zeigt, am hellsten.
D.h. die Kanten, die von diesem Eckpunkt ausgehen, verlieren an Helligkeit in Richtung der anderen beiden Eckpunkte.
Die Helligkeit der Kante, die dem hellsten Eckpunkt gegenüberliegt hängt größtenteils von den beiden anderen Normalen ab.
Sie wird aber wahrscheinlich geringer sein, als die der anderen Kanten.

b) Bei ambientem Licht, wird der gesamten Oberfläche einer Facette derselbe Farbwert zugewiesen (außerdem werden alle Objekt der Szene gleich hell dargestellt).
Die Blickrichtung auf das Objekt ist also egal.
Diffuse Reflexion sendet Licht in alle Richtungen mit gleicher Energie aus.
Die Blickrichtung ist also "relativ" egal:
Solange sich der Beobachter in einer Position befindet, in die reflektiert wird, ist es egal wo genau er steht.
(Er darf aber zum Beispiel nicht hinter dem Objekt [d.h. diametral zum Lichtstrahl] stehen.)
Bei einer rein spekuläeren Beleuchtung wird das Licht nur in eine Richtung reflektiert (es gibt keine Streuung).
Die Position des Beobachters ist also sehr entscheidend:
Entweder er guckt auf des Objekt genau aus der Richtung, in die reflektiert wird und kann es somit sehen.
Oder er sieht es gar nicht, weil überhaupt kein Licht in seine Richtung gesendet wird.

\subsection*{2. Bump Mapping}
% bump = Beule
Bump Mapping wird eingesetzt, um Oberflächen realistischer zu gestalten.
Dazu werden Erhebungen und Senken in die Oberfläche gesetzt.
Dies geschieht allerdings nur virtuell.
D.h. dass das Polygon seine geometrischen Eigenschaften behält.
Um den gewünschten Effekt zu erzielen, werden die Obeflächennormalen vor der Lichtberechnung so modifiziert, als ob die Oberfläche nicht eben wäre.
Der Vorteil des Verfahrens liegt darin, dass sich die virtuellen Unebenheiten wesentlich schneller berechnen lassen, als wenn man wirklich die zu Grunde liegend Geometrie des Poloygons verändern wollte. Dadurch wird das Verfahren für Echtzeitrendering (wie z.B. in Computerspielen) sehr interessant, weil mit verhältnismäßig wenig Aufwand eine große "Detailtiefe" erzeugt werden kann. Die vermeintliche Oberflächenstruktur ändert sich auch, wenn das Licht in der Szene bewegt wird, was zu zusätzlicher Realitätsgetreue führt (anderer Schattenwurf auf der Oberfläche).
Andererseits wird das Objekt nicht wirklich uneben, sodass die Silhoutten und Schatten des gesamten Objekts nach wie vor eben sind. Dies erkennt man vor allem dann, wenn starke Unebenheiten eingefügt wurden.

\end{document}

