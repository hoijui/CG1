\documentclass[a4paper,headings=small]{scrartcl}
\KOMAoptions{DIV=12}

\usepackage[utf8x]{inputenc}
\usepackage{amsmath}
\usepackage{graphicx}
\usepackage{multirow}
\usepackage{listings}

% define style of numbering
\numberwithin{equation}{section} % use separate numbering per section
\numberwithin{figure}{section}   % use separate numbering per section

% instead of using indents to denote a new paragraph, we add space before it
\setlength{\parindent}{0pt}
\setlength{\parskip}{10pt plus 1pt minus 1pt}

\title{Excercise 5 - \emph{Trace my Ray}}
\subtitle{Computer Graphics I - WS12/13}
\author{\textbf{Team 6} (Rolf Schröder [340126], Robin Vobruba [343773], Bernd Loeber)}
\date{\today}

\pdfinfo{%
  /Title    (Computer Graphics I - WS12/13 - Excercise 5 - Trace my Ray)
  /Author   (Team 6: Rolf Schröder [340126], Robin Vobruba [343773], Bernd Loeber)
  /Creator  ()
  /Producer ()
  /Subject  ()
  /Keywords ()
}

% Simple picture reference
%   Usage: \image{#1}{#2}{#3}
%     #1: file-name of the image
%     #2: percentual width (decimal)
%     #3: caption/description
%
%   Example:
%     \image{myPicture}{0.8}{My huge house}
%     See fig. \ref{fig:myPicture}.
\newcommand{\image}[3]{
	\begin{figure}[htbp]
		\centering
		\includegraphics[width=#2\textwidth]{#1}
		\caption{#3}
		\label{fig:#1}
	\end{figure}
}


\begin{document}

\maketitle

\subsection*{1. Whitted ray tracing}
% Whitted ray tracing ermöglicht neben spekularer Reflektion auch spekulare Brechung nach Snell’s Gesetz.
% Berechnen Sie den effektiven Brechungswinkel für Transmission durch 10 cm Wasser.
% (0.5 Punkte)

Der Brechungswinkel hängt von dem Einfallswinkel ab.
Daher wird hier die Formel für den Brechungswinkel hergeleitet.

\begin{align*}
n_W &= 1.33 \quad \text{Wasser-Brechungsindizes} \\
n_L &= 1.000292 \quad \text{Luft-Brechungsindizes} \\
n_1 \sin(\alpha_1) &= n_2 \sin(\alpha_2) \\
\sin(\alpha_1) &= \frac{n_2}{n_1} \sin(\alpha_2) \\
\alpha_1 &= \arcsin(\frac{n_2}{n_1} \sin(\alpha_2)) \\
\alpha_W &= \arcsin(\frac{n_L}{n_W} \sin(\alpha_L)) \\
\alpha_W &= \arcsin(\frac{1.000292}{1.33} \sin(\alpha_L)) \\
\end{align*}

\image{img/brechungswinkel}{0.8}{Lichtbrechung in verschiedenen Medien}

Siehe fig. \ref{fig:img/brechungswinkel}.

Quellen:
\begin{itemize}
\item http://de.wikipedia.org/wiki/Brechungsindex
\item http://de.wikipedia.org/wiki/Snelliussches\_Brechungsgesetz
\end{itemize}


\subsection*{2. Radiometrie}

\subsubsection*{a) Leuchtkraft der Sonne}
% Die Leuchtkraft der Sonne beträgt 3.846 × 10^25 Watt für eine Wellenlänge im sichtbaren Bereich.
% Wie groß ist die mittlere Strahlungsdichte der Sonne?
% Begründen Sie Ihre Antwort.
% (0.5 Punkte)

TODO

\subsubsection*{a) Lichtenergie in Berlin}
% Wie viel Lichtenergie fällt auch eine 1m^2 große Fläche in Berlin in 1 min. am 21. Juni um 12:00 Mittags?
% Fertige hierzu zunächst eine Skizze an.
% (1 Punkt)

\image{img/sonneneinstrahlung}{0.8}{Sonneneinstrahlung}

Siehe fig. \ref{fig:img/sonneneinstrahlung}.
Die Lichtenergie beträgt ca. $700 \frac{W * min}{m^2}$.
Hierbei wird davon ausgegangen,
dass keine Wolken oder Smog/Feinstaub die Sonne verdecken.
Da die Sonne mittags fast orthogonal auf die Fläche scheint,
sind auch keine Schatten zu erwarten.

Quellen:
\begin{itemize}
\item https://de.wikipedia.org/wiki/Sonnenstrahlung
\end{itemize}


\subsection*{3. Ray-Tracing \& Radiosity}
% Beschreiben Sie, wie man Ray Tracing und Radiosity geeignet kombinieren kann.
% (1 Punkt)

Ray Tracing nutzt perfekte Reflektionen und Radiosity geht von perfekter Streuung aus (Lambertstrahler).
Man kann beide nutzen um realistische (zu gewissen Anteilen streuende und reflektierende) Beleuchtung zu erstellen.

Mittels Radiosity ist es möglich diffuses Licht korrekt wiederzugeben.
Dabei werden auch Effekte wie Farbbluten, indirekte Beleuchtung und Schattenverläufe korrekt wiedergegeben. 

Ein Beispiel hierfür wäre die glatte Oberfläche einer Küchenarbeitsplatte.
Hierbei würde ein Großteil der Lichtstrahlen der Deckenlampe auf die Decke reflektiert und ein kleiner Anteil


\subsection*{4. Bidirectional path tracing}
% Bidirectional path tracing wurde von Veach Guibas und Lafortune Willems eingeführt.
% Diskutieren Sie Vor- und Nachteile des Verfahrens.
% (1 Punkte)

\textbf{Vorteile}

Das Integral konvergiert schneller als beim herkömmlichen Path Tracing.
Performanzgewinn durch schnellere Konvergenz ist größer als der Aufwand für das Tracen der zwei Strahlen (Shooting und Gathering).
Sehr exaktes Verfahren. Wird teilweise für Referenzbilder verwendet.
Simuliert Effekte wie weiche Schatten, Kaustiken und Motion Blur ohne Extraaufwand
Basiert nicht auf perfekt diffusen Lambertstrahlern, wie Radiosity, welche es in der Realität nicht gibt.

\textbf{Nachteile}

Um Störungen (Noise) im Bild zu vermeiden eine hohe Anzahl von Strahlen getracet werden.
Obwohl es schneller als normales Path Tracing ist, benötigt Bidirectional Path Tracing viel Rechenzeit.
Pefekte spekuläre Reflektionen können nicht dargestellt werden.


\end{document}
