\documentclass[a4paper,headings=small]{scrartcl}
\KOMAoptions{DIV=12}

\usepackage[utf8x]{inputenc}
\usepackage{amsmath}
\usepackage{graphicx}
\usepackage{multirow}
\usepackage{listings}

% define style of numbering
\numberwithin{equation}{section} % use separate numbering per section
\numberwithin{figure}{section}   % use separate numbering per section

% instead of using indents to denote a new paragraph, we add space before it
\setlength{\parindent}{0pt}
\setlength{\parskip}{10pt plus 1pt minus 1pt}

\title{Excercise 2 - \emph{Visibility}}
\subtitle{Computer Graphics I - WS12/13}
\author{\textbf{Team 6} (Eugen Torkin, Rolf Paul Schroeder, Robin Vobruba)}
\date{\today}

\pdfinfo{%
  /Title    (Computer Graphics I - WS12/13 - Excercise 2 - Visibility)
  /Author   (Team 6: Eugen Torkin, Rolf Paul Schroeder, Robin Vobruba)
  /Creator  ()
  /Producer ()
  /Subject  ()
  /Keywords ()
}

% Simple picture reference
%   Usage: \image{#1}{#2}{#3}
%     #1: file-name of the image
%     #2: percentual width (decimal)
%     #3: caption/description
%
%   Example:
%     \image{myPicture}{0.8}{My huge house}
%     See fig. \ref{fig:myPicture}.
\newcommand{\image}[3]{
	\begin{figure}[htbp]
		\centering
		\includegraphics[width=#2\textwidth]{#1}
		\caption{#3}
		\label{fig:#1}
	\end{figure}
}


\begin{document}

\maketitle


\section{Theorieaufgaben}

\subsection{1. Wieso wird in OpenGL das Clipping am kanonischen Sichtvolumen vorgenommen?}

Bei Clipping muss berechnet werden, ob ein Objekt sich außerhalb oder innerhalb des Sichtvolumens befindet (oder eventuell beides). Dazu werden die Schnittpunkte des Objekts mit dem Sichtvolumen benötigt. Die Berechnung der Schnittpunkte mit dem kanonischen Sichtvolumen (d.h. in einem normierten Würfel, parallel zu den Sichtachsen) ist einfacher als diejenige zur Berechnung der Schnittpunkte mit einer Pyramide (oder einem anderen, beliebigen Sichtvolumen), da der Rechenaufwand geringer ist.

\subsection{2. Entwickeln Sie den Outcode für das kanonische Sichtvolumen. Welche Primitive können im 3D-Fall behandelt werden?}

Das kanonische Sichtvolumen hat 6 Halbräume (oben, unten, links, rechts, davor, dahinter). Deshalb braucht man einen 6-Bit-Outcode. Wir übernehmen die Codierung für 2D und hängen zwei Ziffern für davor und dahinter hinten an. Das fünfte Bit wird auf 1 gesetzt, wenn sich das Objekt vor dem Sichtvolumen befindet; das sechste Bit wird auf ein gesetzt, wenn sich das Objekt hinter dem Sichtvolumen befindet.

Endpunkte vor dem Sichtvolumen:
\begin{verbatim}
100110 | 100010 | 101010
------------------------
000110 | 000010 | 001010
------------------------
010110 | 010010 | 011010
\end{verbatim}

Endpunkte auf Höhe (zwischen zmin und zmax) des Sichtvolumen:
\begin{verbatim}
100100 | 100000 | 101000
------------------------
000100 | 000000 | 001000
------------------------
010100 | 010000 | 011000
\end{verbatim}

Endpunkte hinter dem Sichtvolumen:
\begin{verbatim}
100101 | 100001 | 101001
------------------------
000101 | 000001 | 001001
------------------------
010101 | 010001 | 011001
\end{verbatim}

Im 3D-Fall können mithilfe des Outcodes nicht nur Linien, sondern auch Ebenen behandelt werden.

\subsection{3. Erläutern Sie die Dualität von Ebenen und Punkten im projektiven 3-Raum}
Für Punkte und Ebenen im homogenen 3D-Raum gelten weitestgehend die selben Rechenregeln. Sie sind jeweils komplett durch 3 Werte beschrieben; bzw. durch einen 4-dimensionalen, homogenen Vektor. Bei einem Punkt beschreibt dieser den Ort des Punktes im Raum. Bei einer Ebene gibt die Richtung des Vektors die Normale wieder, und die Länge beschreibt den Abstand vom Ursprung.

Transformation eines Punktes: $X' = P X$

Transformation einer Ebene: $Z' = P^{-T} Z$

\subsection{4. Was ist die Komplexität von Clipping eines einfachen (nicht notwendigerweise konvexen) Polygons an einem konvexen Polygon? Auf welches "Primitiv" bezieht sich hier die Komplexitätsangabe?}

$O(n m)$, wobei $n$ für die Anzahl der Primitiven der Dimension $D - 1$ des konvexen Polygons steht, und $D$ für die Dimensionalität unseres Raumes. Im 3D-Raum ist $n$ also die Anzahl der Ebenen,  aus denen die Oberfläche des konvexen Polygons besteht. $m$ ist die respektive (TODO: was heisst das ??) Anzahl des anderen Polygons. Clipping benötigt linearen Aufwand bezüglich $n m$, weil man jede der $m$ Flächen gegen bei jeder der $n$ Flächen darauf prüfen muss (TODO: watt??), in welchem Halbraum sie liegt.

\subsection{5. Warum wird Backface-Culling am besten im kanonischen Sichtvolumen durchgeführt?}

Im kanonische Sichtvolumen weisen alle Blickstrahlen in dieselbe Richtung (nämlich parallel zur Z-Achse).
Das heißt, dass der Sehstrahl, der auf das Objektseite fällt, der Vektor (0, 0, 1)$^T$ ist.
Somit entfällt die Berechnung des Skalarproduktes, da sein Wert einfach $n_z$ - der z-Komponente der Normalen - entspricht.
Deren Vorzeichen gibt Auskunft darüber, ob die Objektseite sichtbar (negativ) ist oder nicht (positiv).
Das Erkennen von Rückseiten reduziert sich somit auf einen einfachen Vorzeichentest der z-Komponente der Normalen aller Primitiven.

\subsection{6. Transformationsgesetz für einen Normalenvektor}

\subsubsection{a) Betrachten sie eine Ebene wie in Abbildung 2. Charakterisieren Sie den Normalenvektor $n$ mit Hilfe eines beliebigen Vektors $v$ in der Ebene.}

\subsubsection{b) Mit dem Ergebnis der vorhergehenden Teilaufgabe, leiten Sie die Matrix her welche die Transfor-mation des Normalenvektor beschreibt, wenn mit einer beliebigen affinen Transformationsmatrix
T transformiert wird. Betrachten Sie dazu zunächst eine unbekannte affine Transformation wel-
che das Verhalten der Normale beschreibt, und bestimmen Sie dann in welcher Beziehung die
Transformation zur Transformation steht.}

\subsubsection{c) Warum ist das Transformationsgesetz für einen Normalenvektor anders als für einen "normalen" Vektor?}
Nur othogonale Transformationen erhalten die Orthogonalität. Affine Transformationen sind eine Übermenge der othogonalen. Sie ermöglichen also auch die Othogonalität zerstörende Veränderungen. Normalenvektoren müssen aber immer orthogonal zu ihrem Objekt bleiben.

\subsubsection{d) Hält das Ergebnis von Teilaufgabe b) für den Normalenvektor einer Ebene auch für beliebige, möglicherweise gekrümmte Objekte?}
Ja, weil man sich jedes Objekt aus unendlich kleinen ebenen zusammengebaut vorstellen kann.

\end{document}

